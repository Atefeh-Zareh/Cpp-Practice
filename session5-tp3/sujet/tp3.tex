% Created 2021-03-29 lun. 12:00
% Intended LaTeX compiler: pdflatex
\documentclass{article}
\usepackage[utf8]{inputenc}
\usepackage[T1]{fontenc}
\usepackage{graphicx}
\usepackage{grffile}
\usepackage{longtable}
\usepackage{wrapfig}
\usepackage{rotating}
\usepackage[normalem]{ulem}
\usepackage{amsmath}
\usepackage{textcomp}
\usepackage{amssymb}
\usepackage{capt-of}
\usepackage{hyperref}
\usepackage{minted}
\usepackage[english]{babel}
\author{Giuseppe Lipari}
\date{\today}
\title{Conception Objets avancée : TP 3}
\hypersetup{
 pdfauthor={Giuseppe Lipari},
 pdftitle={Conception Objets avancée : TP 3},
 pdfkeywords={},
 pdfsubject={},
 pdfcreator={Emacs 25.2.2 (Org mode 9.1.6)}, 
 pdflang={English}}
\begin{document}

\maketitle

\section*{Introduction}
\label{sec:orgc2ed1f0}

Le but de ce TP est de comprendre la programmation par \emph{templates}.
Nous allons implementer des fonctions pour faire l'union et
l'intersection des deux containers.  

\emph{Attention} : des fonctions similaires sont déjà disponibles dans la
librairie standard : \texttt{std::set\_intersection} et
\texttt{std::set\_union}. Nous utiliserons des noms un peu différents pour
éviter toute confusion.


\subsection*{Définitions}
\label{sec:org7f24694}

L'intersection de deux ensembles \emph{A} et \emph{B} est l'ensemble \emph{C} qui
contient tous les éléments qui sont présents dans le deux ensembles :
\[
       C = A \cap B = { x | x \in A \wedge x \in B}
   \]

L'union de deux ensembles \emph{A} et \emph{B} est l'ensemble \emph{C} qui contient
tous les éléments qui sont présents dans un de deux ensembles:
\[
       C = A \cup B = { x | x \in A \vee x \in B}
   \]
L'ensemble \emph{C} ne contient pas de doublons.

L'objectif \textbf{final} est d'implanter ces deux fonctions de manière la
plus générale possible : les fonctions doivent fonctionner sur
n'importe quel container, et sur n'importe quel type de donnée
contenue.


\section*{Question 1: vecteurs d'entiers}
\label{sec:org25407f2}

Écrire deux fonctions, \texttt{set\_intersection} et \texttt{set\_union} qui, à
partir de deux vecteurs d'entiers, remplissent un vecteur qui
contient l'intersection (l'union, respectivement) de deux vecteurs
d'entiers.

Voici le prototype de la fonction \texttt{set\_intersection\_nt()} :

\begin{minted}[]{c++}
void set_intersection_nt(std::vector<int>::const_iterator a_begin, 
                         std::vector<int>::const_iterator a_end, 
                         std::vector<int>::const_iterator b_begin, 
                         std::vector<int>::const_iterator b_end,
                         std::back_insert_iterator<std::vector<int>> c_begin);
\end{minted}
(\texttt{nt} indique la version non-template). 
La fonction sera utilisé comme dans le programme suivant: 

\begin{minted}[]{c++}
vector<int> vec_a = {1, 2, 3, 4};
vector<int> vec_b = {3, 4, 5, 6};
vector<int> vec_c;

set_intersection_nt(begin(vec_a), end(vec_a), begin(vec_b), end(vec_b), 
                    back_inserter(vec_c));
\end{minted}

La fonction \texttt{set\_union\_nt} a le même prototype et on l'utilise de la
même manière.

Testez les deux fonctions, surtout dans les cas limites (intersection
nulle, l'union de deux ensembles vides, etc.) 

\subsection*{Réponse}
\label{sec:org8329b7f}

\emph{Écrire ici votre réponse, et indiquez les tests que vous avez écrit.}


\section*{Question 2: Généralisation sur les containers}
\label{sec:org9ab3bdf}

Généralisez les fonctions développé dans la question 2 en utilisant
des templates. Il faut donc écrire deux fonctions templates,
\texttt{set\_intersection(...)} et \texttt{set\_union(...)}.  Il y a deux type de
généralisation possible :

\begin{itemize}
\item Type d'objet contenu : par exemple, faire l'intersection entre
deux vecteurs de \texttt{string}.
\item Type de container : par exemple, faire l'union entre deux \texttt{list<int>}.
\end{itemize}

Nous vous demandons de faire \uline{les deux généralisations} au même
temps. En particulier, avec la même fonction template
\texttt{set\_intersection}, il doit être possible de faire l'intersection
entre deux \texttt{vector<string>}, et entre deux \texttt{list<int>}. 

Écrire des tests pour vérifier cela. 

\subsection*{Réponse}
\label{sec:org95e3a92}

\emph{Écrire votre réponse ici, et indiquez les tests que vous avez écrit.} 


\section*{Question 3: Containers d'objets}
\label{sec:org4bbbb08}

Créer de vecteurs et des listes d'objets de type \texttt{MyClass}, et
essayez de faire des intersections et des unions avec les fonctions
développées dans la Question 2. Vérifiez que tout est correct.

Quel sont les caracteristiques minimales de la classe pour
pouvoir faire l'intersection et l'union ?

\subsection*{Réponse}
\label{sec:orgdff6ec8}

\emph{Indiquez les test que vous avez développé}.
\emph{Écrire ici les caractéristiques minimales de la class \texttt{MyClass}}.


\section*{Question 4}
\label{sec:orgdc4e9e1}

Est-ce qu'on peut appeler \texttt{set\_intersection\_t} sur deux containers qui
contiennent objets de type différents ? 

Si non, pourquoi ? Si oui, quel sont le conditions minimales
sur les types des objets, et quel est le résultat finale ?


\subsection*{Réponse}
\label{sec:org01f61df}

\emph{Écrire ici votre réponse}. 


\section*{Question 5}
\label{sec:org25de977}

Essayez d'appliquer vos fonctions sur des \texttt{map<int, string>}. 
Est-ce que ça marche ? Si oui, pourquoi ? Si non, pourquoi ?

\subsection*{Réponse}
\label{sec:org739e169}

\emph{Écrire ici votre réponse}. 


\section*{Question 6}
\label{sec:org211b09c}

\begin{enumerate}
\item Généraliser la fonction \texttt{set\_intersection\_t} avec un paramétre
template additionnel \texttt{F} qui represents une fonction de
comparaison entre les objets des deux containers.

\item Appliquer la fonction \texttt{set\_intersection\_t} sur une \texttt{map<int,
     string>} and sur un \texttt{vector<int>}, le resultat sera produit sur
une \texttt{map<int, string>} :

\begin{minted}[]{c++}
bool my_compare(const std::pair<int, string> &x, int y);

map<int, string> a, b;
vector<int> v = {1, 3, 5};

a[1] = "A"; a[2] = "B"; a[3] = "C"; a[4] = "D";

set_intersection_t(begin(a), end(a), begin(v), end(v), 
                 inserter(b, b.end()), my_compare); 

// b should contain (1, "A") and (3, "C") and nothing else. 
\end{minted}

\begin{enumerate}
\item Testez le résultat.
\end{enumerate}
\end{enumerate}


\subsection*{Réponse}
\label{sec:org3499c4b}

\emph{Réponse dans le code} 
\end{document}